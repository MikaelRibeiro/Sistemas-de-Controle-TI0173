%----------- Importação de pacotes-------------------------------
\documentclass[12pt,a4]{article}
\usepackage[utf8]{inputenc}
\usepackage{graphicx} % Required for inserting images
\usepackage{float}
\usepackage{algorithm}
\usepackage{algorithmic}
\usepackage[portuguese,brazilian,english]{babel}
\usepackage[lmargin=3cm,tmargin=3cm,rmargin=2cm,bmargin=2cm]{geometry} %Formato que lembra a ABNT
\usepackage[T1]{fontenc} %Ajusta o texto que vem de outras fontes
\usepackage{amsmath,amsthm,amsfonts,amssymb,dsfont,mathtools,blindtext} %pacotes matemáticos
\usepackage{graphics,xcolor,enumerate}
\usepackage{fancyhdr}
\usepackage{booktabs}
\usepackage{indentfirst}
\usepackage{animate}
\usepackage{algorithmic}
\usepackage{tikz}
\usetikzlibrary{shapes.geometric, arrows}
\usepackage[colorlinks=true,urlcolor=red,linkcolor=blue]{hyperref}
\usepackage{listings}
\lstdefinestyle{matlabstyle}{
  language=Matlab,
  inputencoding=utf8,
  basicstyle=\ttfamily\footnotesize,
  numbers=left,
  numberstyle=\tiny,
  stepnumber=1,
  numbersep=5pt,
  backgroundcolor=\color{white!95!gray},
  showspaces=false,
  showstringspaces=false,
  showtabs=false,
  frame=single,
  rulecolor=\color{black},
  tabsize=2,
  captionpos=b,
  breaklines=true,
  breakatwhitespace=false,
  title=\lstname,
  keywordstyle=\color{blue},
  commentstyle=\color{red},
  stringstyle=\color{magenta},
  morekeywords={matlabFunction, eig}
}

\lstset{
  inputencoding=utf8,
  literate=
    {á}{{\'a}}1 {é}{{\'e}}1 {í}{{\'i}}1 {ó}{{\'o}}1 {ú}{{\'u}}1
    {â}{{\^a}}1 {ê}{{\^e}}1 {ô}{{\^o}}1
    {ã}{{\~a}}1 {õ}{{\~o}}1
    {ç}{{\c{c}}}1
    {Á}{{\'A}}1 {É}{{\'E}}1 {Í}{{\'I}}1 {Ó}{{\'O}}1 {Ú}{{\'U}}1
    {Â}{{\^A}}1 {Ê}{{\^E}}1 {Ô}{{\^O}}1
    {Ã}{{\~A}}1 {Õ}{{\~O}}1
    {Ç}{{\c{C}}}1
}

\usepackage{hyperref}
\hypersetup{
    colorlinks=true,
    linkcolor=blue,
    urlcolor=blue,
    citecolor=blue
}
%----------------------------------------------------------------
%----------------Configurações do estilo da página---------------
\pagestyle{fancy} % estilo fancyhdr
\fancyhf{} % Limpa todos os campos

% Configuração do nome da sessão
\fancyfoot[L]{\leftmark} % Adiciona o nome da seção atual à esquerda do rodapé
\fancyfoot[R]{\thepage} % Adiciona o número da página à direita do rodapé
% Ajuste da linha horizontal do rodapé
\renewcommand{\footrulewidth}{0.4pt} % Define a espessura da linha
\renewcommand{\headrulewidth}{0pt} % Remove a linha do cabeçalho
\setlength{\footskip}{30pt} % Aumenta a distância do rodapé
%--------------------------------------------------------

\begin{document}
%-------------------------------------------------------------------
%Comandos novos criados para a configuração da página
\begin{titlepage}
\newcommand{\capaUniversidade}{Universidade Federal do Ceará}
\newcommand{\capaCentro}{Centro de Tecnologia}
\newcommand{\capaDepartamento}{Departamento de Engenharia de Teleinformática}
%\newcommand{\capaRelatorio}{Sistemas de Controle} 
\newcommand{\capaAtividade}{Sistemas de Controle}
\newcommand{\capaTitulo}{Project Work: Sistema Hidráulico}
\newcommand{\capaProfessor}{Dr Michela Mulas}
\newcommand{\capaAutorA}{Eduarda Gurgel Dutra Guimarães}
\newcommand{\capaAutorB}{Sezanildo da Silva Paula Filho}
\newcommand{\capaAutorC}{Jordana Nascimento Farias Braga}
\newcommand{\capaAutorD}{Mikael Ribeiro Camelo}
\newcommand{\capaMatriculaA}{512339}
\newcommand{\capaMatriculaB}{499872}
\newcommand{\capaMatriculaC}{514442}
\newcommand{\capaMatriculaD}{509482}
\newcommand{\capaCidade}{Fortaleza}
\newcommand{\capaAno}{2025.1}
\newcommand{\Hrule}{\rule{\linewidth}{0.5mm}}
%-------------------------------------------------------------------

\center
\vspace*{-3.9cm}
\begin{figure}[H] 
    \centering
    \includegraphics[width=0.15\columnwidth]{images.png} % Altere o caminho da imagem
\end{figure}

\large


\textbf{\Large\capaUniversidade}\\
\textbf{\Large\capaCentro}\\
\textbf{\Large\capaDepartamento}\\
%\textbf{\textbf{\Large\capaRelatorio}}\\
\vspace{5cm}
\textbf{\textbf{\Large\capaAtividade}}\\


\vspace{0.5 cm}
\Hrule\\[0.4cm ]
{\huge\textbf{\capaTitulo}}\\[0.25cm]
\Hrule \\[1.0cm]
\vspace{1.0cm}

\textbf{\large Professora: \capaProfessor}\\
\vspace{1.5cm}
\begin{flushleft}
\textbf{\capaAutorA: \capaMatriculaA} \\
\textbf{\capaAutorC: \capaMatriculaC} \\
\textbf{\capaAutorD: \capaMatriculaD} \\
\textbf{\capaAutorB: \capaMatriculaB} \\

\end{flushleft}

\vfill
\textbf{\large\capaCidade}\\ \textbf{\capaAno}\\

\end{titlepage}
%--------------------------------------------------------------

\tableofcontents
\newpage
\section{Sistema hidráulico}
Considere o sistema de dois tanques dispostos como mostrado na Figura:
\begin{figure}[!hbt]
    \centering
    \includegraphics[width=0.5\linewidth]{Imagens/Captura de Tela 2025-06-04 às 21.13.24.png}
    \caption{Sistema de dois tanques }
    \label{fig:enter-label}
\end{figure}

A vazão volumétrica $Q_{\text{in}}$  representa as entradas do sistema, enquanto a variável medida (saída) é o nível de fluido $ h_{2} $ no segundo tanque. As áreas dos dois tanques são $ A_{1} $ e $ A_{2} $, enquanto $a_{1} $ e $ a_{2} $ são as áreas dos orifícios indicadas na Figura~1. O fluido é perfeito (sem tensões de cisalhamento, sem viscosidade, sem condução de calor) e está sujeito apenas à gravidade\textsuperscript{3}. Os tanques estão cheios de água (fluido incompressível) e a pressão externa é constante (pressão atmosférica).
\subsection{Questão 1}

    Obtenha um modelo de espaço de estados do sistema, onde $ x_1(t) = h_1(t) $ e $ x_2(t) = h_2(t) $ são variáveis de estado, $ u(t) = Q_{\text{in}}(t) $  é a entrada e $  y(t) = h_2(t) $  é a saída.\\
    \textit{Resolução}
\subsubsection{Definições e variáveis}

Consideramos o sistema hidráulico de dois tanques ilustrado. As seguintes variáveis são definidas:

\begin{itemize}
  \item $ x_1(t) = h_1(t) $: altura do fluido no tanque 1 (estado)
  \item $ x_2(t) = h_2(t) $: altura do fluido no tanque 2 (estado)
  \item $ u(t) = Q_{in}(t) $: vazão volumétrica de entrada (entrada)
  \item $ y(t) = h_2(t) = x_2(t) $: saída do sistema
  \item $ A_1, A_2 $: áreas das seções transversais dos tanques
  \item $ a_1, a_2 $: áreas dos orifícios de saída
  \item $ g $: aceleração da gravidade
\end{itemize}
\newpage
\subsubsection{Modelagem do sistema}

A variação de volume em cada tanque pode ser modelada como:

\begin{equation}
A_1 \dfrac{dh_1}{dt} = Q_{in} - Q_1,
\end{equation}
\begin{equation}
A_2 \dfrac{dh_2}{dt} = Q_1 - Q_2.
\end{equation}

Com o auxílio  da equação de \textit{Torricelli}, podemos calcular a 
\textbf{vazão volumétrica, que é dada por:}
\begin{equation}
    Q=a\cdot v,
\end{equation}
onde $v$ é a velocidade do escoamento e $a$ a área do orifício. Dessa forma, ficamos com:
\begin{equation}
Q_1 = a_1 \sqrt{2g h_1}, \quad Q_2 = a_2 \sqrt{2g h_2},
\end{equation}
lembrando que $v=\sqrt{2a\Delta s}$ (\textit{Torricelli)}.

Substituindo em $1$ e $2$:

\begin{equation}
A_1 \frac{dh_1}{dt} = Q_{in} - a_1 \sqrt{2g h_1},
\end{equation}
\begin{equation}
A_2 \frac{dh_2}{dt} = a_1 \sqrt{2g h_1} - a_2 \sqrt{2g h_2}.
\end{equation}

\subsubsection{Modelo de Espaço de Estados}

Definindo $ x_1(t) = h_1(t) $, $ x_2(t) = h_2(t) $, $ u(t) = Q_{in}(t) $, temos:

\begin{equation}
\dot{x}_1(t) = \frac{1}{A_1} \left( u(t) - a_1 \sqrt{2g x_1(t)} \right)
\end{equation}
\begin{equation}
\dot{x}_2(t) = \frac{1}{A_2} \left( a_1 \sqrt{2g x_1(t)} - a_2 \sqrt{2g x_2(t)} \right)
\end{equation}
\begin{equation}
y(t) = x_2(t)
\end{equation}

Este é o modelo em espaço de estados  do sistema hidráulico.


    
    %------------------------------------------------
\subsection{Questão 2}
    Dado $ a_1 = 0{,}02\,\text{m}^2 $, $a_2 = 0{,}05\,\text{m}^2 $, $ A_1 = 0{,}5\,\text{m}^2 $ e $ A_2 = 2\,\text{m}^2 $, encontre o estado de equilíbrio $ (\bar{x}_1, \bar{x}_2) $ obtido com uma entrada constante $ \bar{u} = 0{,}5\,\text{m}^3/\text{s} $, aproximando a aceleração da gravidade como $g = 9{,}81\,\text{m}/\text{s}^2 $.\\
    \textit{Resolução}
\subsubsection{Condição de Equilíbrio}
    No estado de equilíbrio, as derivadas temporais são nulas $(\dot{x}_1 = 0, \dot{x}_2 = 0)$. Substituindo nas equações do espaço de estado (7) e (8):

    \subsubsection{Para $\dot{x}_1 = 0:$}

     \begin{equation}
    0 = \frac{1}{A_1} (\bar{u} - a_1 \sqrt{2g \bar{x}_1} \Longrightarrow \bar{u} = a_1\sqrt{2g \bar{x}_1}
   \end{equation}

    Resolvendo para $\bar{x}_1:$

     \begin{equation}
    \bar{x}_1 = \frac{\bar{u}^2}{2ga_1^2} = \frac{0,5^2}{2 \times 9,81 \times 0,005^2} = \frac{0,25}{0,007848} \approx 31,86m
    \end{equation}

   \subsubsection{Para $\dot{x}_2 = 0:$}
    
     \begin{equation}
    0 = \frac{1}{A_2} (a_1 \sqrt{2g \bar{x}_1} - a_2 \sqrt{2g \bar{x}_2}) \Longrightarrow a_1\sqrt{2g \bar{x}_1} = a_2 \sqrt{2g \bar{x}_2}
    \end{equation}

    Simplificando e substituindo $a_1\sqrt{2g \bar{x}_1} = \frac{\bar{u}}{a_1}:$

     \begin{equation}
    \bar{u} =  a_2 \sqrt{2g \bar{x}_2} \Longrightarrow \bar{x}_2 = \frac{\bar{u}^2}{2ga_2^2} = \frac{0,5^2}{2 \times 9,81 \times 0,05^2} = \frac{0,25}{0,04905} \approx 5,10m
    \end{equation}

    \subsubsection{Resultado Final}

    O estado de equilíbrio é:

     \begin{equation}
    (\bar{x}_1, \bar{x}_2) \approx (31,86m; 5,10m)
    \end{equation}
    
%------------------------------------------------------------
\subsection{Questão 3}
    Determine o sistema linearizado $ (\mathbf{A,B,C,D}) $ em torno do estado de equilíbrio.\\
    \textit{Resolução}

    \subsubsection*{Ponto de equilíbrio (já obtido na Questão 2)}
    \begin{equation}
\bar{x}_1 = 31{,}89 \, \text{m}, \quad \bar{x}_2 = 5{,}10 \, \text{m}
    \end{equation}
\subsubsection*{Linearização:}
Utilizando o método de expansão em série de Taylor de primeira ordem, obtemos:

 \begin{equation}
\delta \dot{x} = A \delta x + B \delta u, \quad \delta y = C \delta x + D \delta u
 \end{equation}

\subsubsection*{Matriz A: derivadas parciais de \(\dot{x}_1, \dot{x}_2\) em relação a \(x_1, x_2\)}

\begin{equation}
A = \begin{bmatrix}
\dfrac{-a_1 g}{A_1 \sqrt{2g \bar{x}_1}} & 0 \\
\dfrac{a_1 g}{A_2 \sqrt{2g \bar{x}_1}} & \dfrac{-a_2 g}{A_2 \sqrt{2g \bar{x}_2}}
\end{bmatrix}
=
\begin{bmatrix}
-0{,}0157 & 0 \\
0{,}00392 & -0{,}0245
\end{bmatrix}
\end{equation}

\subsubsection*{Matriz B: derivadas parciais em relação à entrada \(u\)}

\begin{equation}
B = \begin{bmatrix}
\dfrac{1}{A_1} \\
0
\end{bmatrix}
=
\begin{bmatrix}
2{,}0 \\
0
\end{bmatrix}
\end{equation}

\subsubsection*{Matriz C: derivadas parciais da saída em relação aos estados}

\begin{equation}
C = \begin{bmatrix} 0 & 1 \end{bmatrix}
 \end{equation}

\subsubsection*{Matriz D: derivadas parciais da saída em relação à entrada}

\begin{equation}
D = \begin{bmatrix} 0 \end{bmatrix}
 \end{equation}

\subsubsection*{Conclusão}

O sistema linearizado na forma espaço de estados é dado por:

\begin{equation}
\begin{aligned}
A &= \begin{bmatrix}
-0{,}0157 & 0 \\
0{,}00392 & -0{,}0245
\end{bmatrix}, \quad
B = \begin{bmatrix}
2{,}0 \\
0
\end{bmatrix}, \\
C &= \begin{bmatrix} 0 & 1 \end{bmatrix}, \quad
D = \begin{bmatrix} 0 \end{bmatrix}
\end{aligned}
 \end{equation}
Este modelo aproxima o comportamento dinâmico do sistema ao redor do ponto de operação \((\bar{x}_1, \bar{x}_2)\) de forma linear, permitindo análises e projetos de controle mais simples.
%------------------------------------------------------------
\subsection{Questão 4}
Na linguagem de programação de sua escolha, simule e compare a resposta dos dois sistemas a um degrau de entrada com amplitude $ A_d $ de sua escolha. Discuta e comente.


\subsection*

Para resolver essa questão, foram utilizadas as informações obtidas nas Questões 2 e 3, incluindo o ponto de equilíbrio $(\bar{x}_1, \bar{x}_2) = (31{,}86\,\text{m},\,5{,}10\,\text{m})$ e o sistema linearizado com as seguintes matrizes:

\begin{align*}
A &= \begin{bmatrix}
-0{,}0157 & 0 \\
0{,}0157 & -0{,}0245
\end{bmatrix}, \quad
B = \begin{bmatrix}
2 \\
0
\end{bmatrix}, \quad
C = \begin{bmatrix}
0 & 1
\end{bmatrix}, \quad
D = 0.
\end{align*}

A simulação foi realizada no \texttt{MATLAB}, considerando:
\begin{itemize}
    \item Parâmetros do sistema: $a_1 = 0{,}02\,\text{m}^2$, $a_2 = 0{,}05\,\text{m}^2$, $A_1 = 0{,}5\,\text{m}^2$, $A_2 = 2\,\text{m}^2$, $g = 9{,}81\,\text{m/s}^2$;
    \item Entrada constante no equilíbrio: $\bar{u} = 0{,}5\,\text{m}^3/\text{s}$;
    \item Degrau de entrada aplicado: $u(t) = 0{,}6\,\text{m}^3/\text{s}$;
    \item Tempo de simulação: $t \in [0,\,20]\,\text{s}$.
\end{itemize}

\subsubsection*{Modelo Não Linear}
O sistema não linear é descrito pelas equações diferenciais:

\begin{align*}
\dot{x}_1(t) &= \frac{1}{A_1}\left(u(t) - a_1\sqrt{2g x_1(t)}\right), \\
\dot{x}_2(t) &= \frac{1}{A_2}\left(a_1\sqrt{2g x_1(t)} - a_2\sqrt{2g x_2(t)}\right).
\end{align*}

A simulação foi feita com \texttt{ode45} a partir do ponto de equilíbrio $(x_1(0), x_2(0)) = (\bar{x}_1, \bar{x}_2)$.

\subsubsection*{Modelo Linearizado}

O modelo linearizado foi implementado com o sistema em espaço de estados $(A, B, C, D)$ e simulado com a entrada linearizada $u_{\text{lin}}(t) = u(t) - \bar{u}$ e estado inicial nulo $\delta x(0) = 0$.

\subsubsection*{Conclusão}

Os resultados são apresentados nos gráficos, comparando a altura $h_2(t)$ (saída) do modelo linearizado com a do modelo não linear. Observa-se que o modelo linearizado aproxima bem o comportamento do sistema para pequenas variações em torno do ponto de equilíbrio, mas diverge à medida que a entrada se afasta do valor nominal $\bar{u}$.

\begin{itemize}
    \item O modelo não linear apresenta uma resposta mais realista por considerar a dinâmica exata da vazão (lei de Torricelli).
    \item O modelo linearizado é útil para análise de estabilidade, controlabilidade e projeto de controladores lineares.
\end{itemize}

\begin{figure}[!hbt]
    \centering
    \includegraphics[width=0.7\linewidth]{Imagens/evolucao_h1_nlinear.png}
    \caption{Evolução do modelo não linear.}
    \label{fig:enter-label}
\end{figure}

\begin{figure}[!hbt]
    \centering
    \includegraphics[width=0.7\linewidth]{Imagens/comparacao_h2_modelos.png}
    \caption{Comparação da saída.}
    \label{fig:enter-label}
\end{figure}

\newpage
\section{Matlab Codes}
\begin{lstlisting}[style=matlabstyle]
clc;
clear;
close all;

%% PARÂMETROS DO SISTEMA
g = 9.81;         % gravidade      
a1 = 0.02;        % área do orifício     
a2 = 0.05;        %área do orifício    
A1 = 0.5;         % Área da seção transversal     
A2 = 2;           % área da seção transversal     
u0 = 0.5;         %  entrada de equilibrio
x1_eq = 31.86;    % ponto de equilíbrio
x2_eq = 5.10;     % ponto de equilíbrio

%% TEMPO DE SIMULAÇÃO
tspan = [0 20]; % segundos
u_step = 0.6;   % degrau de entrada (amplitude escolhida maior que u0)

%% 1. MODELO NÃO-LINEAR (ODE)
nonlinear_ode = @(t, x) [ ...
    (1/A1)*(u_step - a1*sqrt(2*g*x(1))); ...
    (1/A2)*(a1*sqrt(2*g*x(1)) - a2*sqrt(2*g*x(2))) ...
];

x0_nl = [x1_eq; x2_eq]; % estado inicial no ponto de equilíbrio
[t_nl, x_nl] = ode45(nonlinear_ode, tspan, x0_nl);

%% 2. MODELO LINEARIZADO
% Matrizes obtidas na questão 3
A = [-0.0157,  0;
      0.0157, -0.0245];

B = [2;
     0];

C = [0 1];
D = 0;

% Sistema linearizado
sys_linear = ss(A, B, C, D);

% Estado inicial em zero (devido a linearização em torno do equilibrio)
x0_lin = [0; 0]; 

% Entrada linearizada: degrau = (u_step - u0)
u_lin = (u_step - u0) * ones(size(t_nl));

[y_lin, t_lin, x_lin] = lsim(sys_linear, u_lin, t_nl, x0_lin);
y_nl = x_nl(:,2); % saída do modelo não linear

%% PLOTAGEM
figure;

subplot(2,1,1);
plot(t_nl, y_nl, 'b', 'LineWidth', 1.8); hold on;
plot(t_lin, x2_eq + y_lin, 'r--', 'LineWidth', 1.8);
grid on;
xlabel('Tempo (s)');
ylabel('Altura h_2 (m)');
title('Resposta dos Sistemas ao Degrau na Entrada');
legend('Modelo Não-linear', 'Modelo Linearizado');

subplot(2,1,2);
plot(t_nl, x_nl(:,1), 'k', 'LineWidth', 1.5);
xlabel('Tempo (s)');
ylabel('Altura h_1 (m)');
grid on;
title('Estado x_1 (h_1) no Modelo Não-linear');



\end{lstlisting}
\end{document}
